\cvsection{Cursos de posgrado}


\begin{itemize}
\item {SIG y Teledetección en Sistemas Agropecuarios}
\item{Sensores remotos aplicados al estudio de ecosistemas agrícolas y naturales}
\item{Aplicaciones de la teledetección óptica y radar de apertura sintética en las ciencias agrarias y ambientales}
\item{Análisis de modelos de elevación}
%\item{Análisis de modelos de elevación (MDE): Geomorfometría}
\item{Reconocimiento, clasificación de suelos y propiedades diagnósticas para su uso y manejo}
\item{Indicadores de calidad de suelos}
%\item{Génesis y Evolución de Suelos}
%\item{Suelos y Geomorgología}
%\item{Fundamentos de la geología del cuaternario}
\item{Bioestadística, geoestadística y diseño experimental}
%\item{Análisis de Datos Multivariados}
%\item{Diseño experimental y análisis de la varianza}
%\item{Geoestadística}
\end{itemize}

\divider

\cvsideevent{Cursos del "Módulo ASI"}{ Università degli studi di Pavia, Italia. 2023}{}{}{}{}

\begin{itemize}
\item{Mapping soil moisture and inundations at high resolution} 
\item{Retrieval of biophysical parameters from optical and radar data} 
\item{Processing of SAR images: integration of multi-frequency polarimetric SAR and optical data for sensitivity analysis
and creation of thematic maps} 
\item{Hyperspectral data processing} 
\item{Machine learning for Earth Observation} 
\item{Earth Engine for EO data processing} 
\end{itemize}


\divider


\cvsideevent{Cursos de la Maestría en Aplicaciones de Información Espacial}{Instituto Gulich (CONAE-UNC) 2022}{}{}{}{}
\begin{itemize}
%\item {Matemática}
%\item {Estadística }
\item {Introducción a la teledetección}
%\item {Introducción a las técnicas inteligentes de resolución de problemas de planificación, secuenciación y ejecución}
\item {Programación y métodos numéricos orientados al tratamiento de información satelital}
\item {Teledetección de recursos agrícolas y forestales}
\item {Procesamiento digital de imágenes satelitales y SIG}
\item {Aplicaciones de imágenes de radar de apertura sintética}
\item {Modelos numéricos de alerta temprana, mapas de riesgo y simulación}
\item {Análisis espacial y situaciones de riesgo}
\item {Herramientas de evaluación, monitoreo y gestión ambiental}
%\item {Metodologías de la investigación}
\end{itemize}