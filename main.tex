%%%%%%%%%%%%%%%%%
% This is an sample CV template created using altacv.cls
% (v1.1.5, 1 December 2018) written by LianTze Lim (liantze@gmail.com). Now compiles with pdfLaTeX, XeLaTeX and LuaLaTeX.
%
%% It may be distributed and/or modified under the
%% conditions of the LaTeX Project Public License, either version 1.3
%% of this license or (at your option) any later version.
%% The latest version of this license is in
%%    http://www.latex-project.org/lppl.txt
%% and version 1.3 or later is part of all distributions of LaTeX
%% version 2003/12/01 or later.
%%%%%%%%%%%%%%%%

%% If you need to pass whatever options to xcolor
\PassOptionsToPackage{dvipsnames}{xcolor}
\PassOptionsToPackage{lowtilde}{url}

%% If you are using \orcid or academicons
%% icons, make sure you have the academicons
%% option here, and compile with XeLaTeX
%% or LuaLaTeX.
% \documentclass[10pt,a4paper,academicons]{altacv}

%% Use the "normalphoto" option if you want a normal photo instead of cropped to a circle
% \documentclass[10pt,a4paper,normalphoto]{altacv}

\documentclass[9.5pt,a4paper,ragged2e, normalphoto]{altacv}

%% AltaCV uses the fontawesome and academicon fonts
%% and packages.
%% See texdoc.net/pkg/fontawecome and http://texdoc.net/pkg/academicons for full list of symbols. You MUST compile with XeLaTeX or LuaLaTeX if you want to use academicons.

% Change the page layout if you need to
\geometry{left=1cm,right=9cm,marginparwidth=6.5cm,marginparsep=1cm,top=1cm,bottom=1cm}

% Change the font if you want to, depending on whether
% you're using pdflatex or xelatex/lualatex
\ifxetexorluatex
  % If using xelatex or lualatex:
  \setmainfont{Carlito}
\else                                                                                  
  \usepackage[utf8]{inputenc}
  \usepackage[T1]{fontenc}
  \usepackage[default]{lato}
\fi
\usepackage{graphicx}
\graphicspath{ {images/} }                 
\definecolor{IFBlue}{HTML}{0F6688}
\definecolor{IFDarkBlue}{HTML}{00425C}
\definecolor{IFMediumBlue}{HTML}{17739A}
\definecolor{IFLightBlue}{HTML}{228FBD}

\definecolor{IFGrey}{HTML}{6D6E71}
\definecolor{IFDarkGrey}{HTML}{231F20}
\definecolor{IFMediumGrey}{HTML}{4D4D4F}
\definecolor{IFLightGrey}{HTML}{808285}

\colorlet{heading}{IFBlue}
\colorlet{accent}{IFLightBlue}
\colorlet{emphasis}{IFMediumGrey}
\colorlet{body}{IFGrey}

% Change the bullets for itemize and rating marker
% for \cvskill if you want to
\renewcommand{\itemmarker}{{\small\textbullet}}
\renewcommand{\ratingmarker}{\faCircle}

%% sample.bib contains your publications
\addbibresource{sample.bib}

% \usepackage{hyperref}

\begin{document}
\name{Emmanuel Leizica}
\tagline{Ingeniero en Recursos Naturales y Medio Ambiente
Maestrando en Aplicaciones de Información Espacial }

\photo{3cm}{logo}
\personalinfo{%
  % Not all of these are required!
  % You can add your own with \printinfo{symbol}{detail}
  Nº Matricula Nacional SAPROCEA: 834-224
  
  \email{emmanuelleizica@gmail.com}
  \phone{+54 9 2954-513538}
  \location{La Pampa, Argentina}
  
  %Fecha de nacimiento: 17/10/1986 
 % \homepage{\url{http://fma.if.usp.br/~nickolas}}
  %% You MUST add the academicons option to \documentclass, then compile with LuaLaTeX or XeLaTeX, if you want to use \orcid or other academicons commands.
  % \orcid{orcid.org/0000-0000-0000-0000}
}

%% Make the header extend all the way to the right, if you want.
\begin{fullwidth} 
\makecvheader
\end{fullwidth}

%% Depending on your tastes, you may want to make fonts of itemize environments slightly smaller
% \AtBeginEnvironment{itemize}{\small}

%% Provide the file name containing the sidebar contents as an optional parameter to \cvsection.
%% You can always just use \marginpar{...} if you do
%% not need to align the top of the contents to any
%% \cvsection title in the "main" bar.
\cvsection[page1sidebar]{Formación académica}

\cvevent{Maestría en Aplicaciones de Información Espacial}{Instituto Gulich (CONAE\footnote{Comisión Nacional de Actividades Espaciales}-UNC\footnote{Universidad Nacional de Córdoba}) 2022 - 2024}{}{}{}{}
%\begin{itemize}
% \item Tesis (en desarrollo): Aplicación de datos geoespaciales derivados de sensores remotos en la estimación de la condición de suelos de uso agrícola.
 %\item Advisor: Mgtr. Salari, Francisco
 %\item Supported (August 2019 - Ongoing) by São Paulo Research Foundation (FAPESP) grant 2019/12158-4
%\end{itemize}

%\divider

\cvevent{Diplomatura en geomática aplicada a la producción agropecuaria}{Instituto Gulich ({CONAE}{ - UNC}) 2019}{}{}{}{}
%\begin{itemize}
 %\item Finalizado
 %\end{itemize}

%\divider

\cvevent{Doctorado en Ciencias Geológicas}{Facultad de Ciencias Exactas, Fisico-Químicas y Naturales (UNRC\footnote{Universidad Nacional de Río Cuarto}) 2016 - 2022}{}{}{}{}
%\begin{itemize}
 %\item Tesis (en desarrollo): Procesos pedogenéticos y mapeo de suelos en el paisaje de meseta de la región semiáreda pampeana, Argentina.
 %\item Advisor: Dra. Elke Noellemeyer
 %\item Supported (August 2019 - Ongoing) by São Paulo Research Foundation (FAPESP) grant 2019/12158-4
%\end{itemize}

%\divider

\cvevent{Ingeniería en Recursos Naturales y Medio Ambiente}{Universidad Nacional de La Pampa 2014}{}{}{}{}
%\begin{itemize}
%\item Tesis: Eficiencia de tratamiento de aguas residuales urbanas.
%\item Finalizado
 %\item Advisors: Maria Daniela Gomez - Maria Cecilia Provensal
 %\item Supported (August 2019 - Ongoing) by São Paulo Research Foundation (FAPESP) grant 2019/12158-4
%\end{itemize}


\cvsection{Experiencia laboral}
\cvevent{Investigador en Sistemas de Información Geográfica aplicado al Agro}{AGSUS S.R.L}{Año: 2021}{La Pampa, Argentina}{}{}
\begin{itemize}
\item Elaboración mapas aptitud del suelo para el secuestro de carbono en establecimientos agrícolas y ganaderos mediante el uso de datos de satélites y plataforma Google Earth Engine y software QGIS.

 \end{itemize}
\divider

\cvevent{Analista ambiental y SIG}{Pequímica Comodoro Rivadavia S.A.}{Año: 2015}{La Pampa, Argentina}{}{}
 \begin{itemize}
 \item Elaboración de documentos de Declacración de Jurada Ambiental (DJI) en el área de Exploración y Explotación de Hidrocarburos en el yacimiento "El Medanito", Departamento Puelén - Provincia de La Pampa.
 \end{itemize}
\divider

\cvevent{Asistente técnico}{Dirección de Recursos Naturales, provincia de La Pampa}{2014 - 2015}{La Pampa, Argentina}{}{}
 \begin{itemize}
\item Monitoreo, control y seguimiento de planes experimentales, de conservación y manejo sostenible, de cambio de uso del suelo y toda otra actividad inherente al bosque nativo, acorde a las leyes forestales.
\item Análisis técnico de los expedientes de infracción a las leyes forestales y llevar al día los registros que se llevan a cabo en el área forestal, acorde a las leyes forestales.
\item Participación en capacitaciones y evaluaciones de Planes de Quemas Prescritas como herramienta de manejo de los sistemas naturales en la provincia de La Pampa. 
\end{itemize}
\divider


%\begin{fullwidth}
\newpage
\cvsection[page2sidebar]{Actividad de investigación}


 % \homepage{\url{http://fma.if.usp.br/~nickolas}}
  %% You MUST add the academicons option to \documentclass, then compile with LuaLaTeX or XeLaTeX, if you want to use \orcid or other academicons commands.
  % \orcid{orcid.org/0000-0000-0000-0000}

\cvevent{}{Participación en proyectos de investigación}{}{}{}{}
\begin{itemize}
 \item {Servicios ecosistémicos en agro-ecosistemas de la región semiárida pampeana. 
 
Universidad Nacional de La Pampa. 01/2020 - 12/2024}

\item {Desarrollo de indicadores para el monitoreo de la calidad en los suelos en la Región Semiárida Pampeana. 
 
Universidad Nacional de La Pampa. 08/2017 - 08/2019}

\item {Evaluación y valoración de los servicios ecosistémicos en la región semiárida pampeana. 
 
Agencia Nacional de Promoción Científica y Tecnológica. 07/2017 - 07/2020}

\item {Manejo sitio-específico de la tierra para mejorar la eficiencia de uso de recursos y el nivel de los servicios ecosistémicos. 
 
Facultad de Agronomía, UNLPam. 01/2015 - 12/2019}

\item {Procesos pedogenéticos y su relación con características de la vegetación en las mesetas de la subregión de mesetas y valles de la región semiárida pampeana, Argentina. 
 
Universidad Nacional de La Pampa. 01/2018 - 12/2021}

\end{itemize}

\cvevent{}{Participación en proyectos de extensión}{}{}{}{}

\begin{itemize}
 \item {Los sistemas de información geográfica como herramienta para el uso del suelo con criterios de sustentabilidad. 
 
Universidad Nacional de La Pampa. 08/2021-08/2023}

\item {Fortalecimiento a las capacidades institucionales para mitigar el efecto de las inundaciones en La Pampa. 
 
Universidad Nacional de La Pampa. 03/2018-03/2020}

\item {Mitigación de las inundaciones. 
 
Ministerio de Educación, Cultura, Ciencia y Tecnología de la Nación. 02/2018-02/2019}
\end{itemize}

\cvevent{}{Publicaciones en revistas científicas}{}{}{}{}
\begin{itemize}
\item Farrell, M., \textbf{Leizica, E}., Gili, A., Noellemeyer, E. Identification of management zones with different potential moisture availability for sustainable intensification of dryland agriculture. 2023. Precision Agriculture, volume 24, pages 1116–1131. DOI:10.1007/s11119-023-10002-2.

\item Morazzo, G., Riestra, D. R., \textbf{Leizica, E}., Álvarez, L., Noellemeyer, E. Afforestation with different tree species causes a divergent evolution of soil profiles and properties. 2021. Front. For. Glob. Change. DOI: 10.3389/ffgc.2021.685827.

\item \textbf{Leizica, E}., Frank Buss, M. E., Noellemeyer, E. Geomorphology as a tool to digitize homogeneous management zones based on soil properties in the semiarid central Argentinean Pampas. 2022. Geoderma Regional. 
https://doi.org/10.1016/j.geodrs.2021.e00458

\item Frank Buss, M. E., \textbf {Leizica, E}., Peinetti, R., Noellemeyer, E. Relationships between landscape features, soil properties, and vegetation determine ecological sites in a semiarid savanna of central Argentina. 2019. Journal of Arid Environments.
doi:10.1016/j.jaridenv.2019.104038.

\end{itemize}
\newpage
\marginpar{
\cvsection{Software/Herramientas}
 \begin{itemize}
 \item Python (librerías geoespaciales)
 \item R
 \item QGIS
 \item GRASS GIS
 \item Google Earth Engine
 \item SNAP
\end{itemize}

\cvsection{Idiomas}
 \begin{itemize}
 \item Inglés
\end{itemize}


\cvsection{Referencias}

\textbf{Mg. Santiago Seppi}

Director de la Maestría en Aplicaciones

de Información Espacial – Instituto Gulich

\phone{+54 9 351 647-4020}

\textbf{Dra. Elke Johana Noellemeyer}

Prof. Facultad de Agronomía UNLPam,

Cátedras de suelo y de Sistema de Información Agro-geográfica

\phone{+54 9 2954 45-1600}

}


\cvevent{}{Trabajos presentados en congresos y jornadas}{}{}{}{}
\begin{itemize}
 \item {Estudio de los incendios ocurridos en la provincia de Misiones, Argentina entre diciembre del año 2021 a marzo del año 2022.
 XXXVI Jornadas Forestales}
\cvevent{} {}
{Mayo 2023}{Entre Ríos, Argentina}{}{}

\item {Estados ecológicos de referencia en la eco-región del caldenal. VIII Congreso Nacional de Manejo de Pastizales Naturales y IV Congreso Internacional de Manejo de Pastizales Naturales}
\cvevent{} {}
{Mayo 2018}{La Rioja, Argentina}{}{}

\item {Relación suelo-paisaje en la subregión de mesetas y valles, La Pampa. III Jornadas Nacionales de Suelos de Ambientes Semiáridos y II Taller Nacional de Cartografía Digital }
\cvevent{} {}
{Septiembre 2017}{Buenos Aires, Argentina}{}{}

\item {Caracterización de sitios ecológicos en la subregión de mesetas y valles, La Pampa. III Jornadas Nacionales de Suelos de Ambientes Semiáridos y II Taller Nacional de Cartografía Digital}
\cvevent{} {}
{Septiembre 2017}{Buenos Aires, Argentina}{}{}

\item {Identificación y mapeo de unidades geomorfológicas de la subregión mesetas y valles de la provincia de La Pampa. XX Congreso Geológico Argentino }
\cvevent{} {}
{2017}{Tucumán, Argentina}{}{}

\item {Dinámica de la cobertura y uso del suelo en los últimos 50 años en el sector oeste de la subregión de los valles transversales de la Pampa. II Jornadas de Suelos de Ambientes Semiáridos}
\cvevent{} {}
{Mayo 2015}{La Pampa, Argentina}{}{}

\end{itemize}

\cvsection{Otros cursos y capacitaciones}

\begin{itemize}
\item {Curso: Introducción a la teledetección. Unidad de Educación y Formación Masiva (UEFM) de la Comisión Nacional de Actividades Espaciales (CONAE). Junio 2020.}

\item {Curso: Satellite Remote Sensing for Agricultural Applications. NASA’s Applied Remote Sensing Training Program (ARSET). Abril 2020.}

%\item {Curso: Elaboración de módulos temáticos 2Mp y su inclusión en propuestas de enseñanza. Programa 2Mp de la Comisión Nacional de Actividades Espaciales (CONAE). Mayo 2020.} 

\item {Curso: Understanding Phenology with Remote Sensing. NASA´s Applied Remote Sensing Training Program (ARSET). Junio y Julio 2020.}

\item {Curso: Investigating Time Series of Satellite Imagery. NASA’s Applied Remote Sensing Training Program (ARSET). Abril 2019.}

\item {Workshop: ESA/CONAE C/L/X band SAR. Organizado por la Agencia Espacial Europea (ESA) y la Comisión Nacional de Actividades Espaciales (CONAE). Noviembre 2018.} 

\item {Curso: Introducción a la teledetección SAR. Unidad de Educación y Formación Masiva (UEFM) de la Comisión Nacional de Actividades Espaciales (CONAE). Agosto 2018.}

\item {Workshop: Escuela de Primavera en Teledetección. El radar SAR como herramienta de monitoreo del ambiente y la producción. Organizado por el Centro Latinoamericano de Formación Interdisciplinaria (CELFI) y el Instituto Mario Gulich (CONAE/UNC). Setiembre 2018.}

\item {Workshop: Biología de la conservación en el manejo de los bosques nativos aplicado a las regiones de Monte y Espinal. Julio 2014.}

\item {Curso: Metodología para la caracterización de sitios ecológicos. Facultad de Agronomía de la Universidad Nacional de La Pampa. Noviembre 2013.}
\end{itemize}




%\cvevent{University Management}{}{}{}{}{}
%\begin{itemize}
 %\item {Coordinator of the advisory committee on workplace and gender violence. Gulich Institute (UNC - CONAE). 
 
 
 %(May 2023-- May 2024)}
  %\item {Member of the board of directors of National University of Río Cuarto.
  
  
  %(School of Exact, Physical-Chemical and Natural Sciences: August 2019 - May 2021; University: August 2021 - February 2022)}
%\end{itemize}

%\end{fullwidth}



\end{document}
